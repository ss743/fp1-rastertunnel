\documentclass[12pt,listof=totoc]{scrartcl}

\usepackage{fancyhdr}
\usepackage{geometry}
\usepackage{ucs}
\usepackage[utf8x]{inputenc}
\usepackage[T1]{fontenc}
\usepackage[ngerman]{babel}
\usepackage{amsmath,amssymb,amstext}
\usepackage{hyperref}
\usepackage{cancel}
\usepackage{dsfont}
\usepackage{physics}
\usepackage{lmodern}
\usepackage{enumerate}
\usepackage{enumitem}
\usepackage{graphicx}
\usepackage{listings, color}
\usepackage[labelfont=bf]{caption}
\usepackage{titling}

\lstset{basicstyle=\scriptsize} %Quellcode mit Umlauten und ganz klein
\lstset{literate=
  {Ö}{{\"O}}1
  {Ä}{{\"A}}1
  {Ü}{{\"U}}1
  {ß}{{\ss}}2
  {ü}{{\"u}}1
  {ä}{{\"a}}1
  {ö}{{\"o}}1
}


%Geometrie----------------------------------------------------------------------------------------------------------

\geometry{a4paper, top=25mm, left=15mm, right=15mm, bottom=25mm,headsep=10mm, footskip=10mm}
\pagestyle{fancy}
\setlength{\parindent}{0pt} %Zeileneinrückung

\fancyhf{} %Setzt voreingestellte Kopf-und Fußzeilen-Eigenschaften zurück

\lhead{\nouppercase{\leftmark}}
\chead{}
\rhead{\thepage}

\lfoot{}
\cfoot{}
\rfoot{}

\title{\vspace{0cm}{\Huge Fortgeschrittenen-Praktikum I:\\ \vspace{1cm} Rastertunnelmikroskop}}
\author{Saskia Bondza\\Simon Stephan}
\date{durchgeführt am 07. und 10.10.2016}

\pretitle{%
  \begin{center}
  \LARGE
  \includegraphics[width=6cm,]{figures/siegel}\\[\bigskipamount]
}
\posttitle{\end{center}}

%neue Commands----------------------------------------------------------------------------------------------------------
\newcommand{\nab}{\vec{\nabla}} %direkter Befehl mit Vektorpfeil
\newcommand{\gra}[3][0.7]{
	\begin{minipage}[h!]{\textwidth}
		\centering
		\includegraphics[width=#1\textwidth]{figures/#2.png}
		\captionof{figure}{#3}
	\end{minipage}
	\vskip 30 pt
}
\newcommand{\graTwo}[4][0.49]{
	\begin{minipage}[h!]{\textwidth}
		\centering
		\includegraphics[width=#1\textwidth]{figures/#2.png}
		\includegraphics[width=#1\textwidth]{figures/#3.png}
		\captionof{figure}{#4}
	\end{minipage}
	\vskip 30 pt
}
\newcommand{\graThree}[5][0.49]{
	\begin{minipage}[h!]{\textwidth}
		\centering
		\includegraphics[width=#1\textwidth]{figures/#2.png}
		\includegraphics[width=#1\textwidth]{figures/#3.png}\newline
		\includegraphics[width=#1\textwidth]{figures/#4.png}
		\captionof{figure}{#5}
	\end{minipage}
	\vskip 30 pt
}
\newcommand{\graFive}[7][0.49]{
	\begin{minipage}[h!]{\textwidth}
		\centering
		\includegraphics[width=#1\textwidth]{figures/#2.png}
		\includegraphics[width=#1\textwidth]{figures/#3.png}\newline
		\includegraphics[width=#1\textwidth]{figures/#4.png}
		\includegraphics[width=#1\textwidth]{figures/#5.png}
		\includegraphics[width=#1\textwidth]{figures/#6.png}\newline
		\captionof{figure}{#7}
	\end{minipage}
	\vskip 30 pt
}
\newcommand{\del}[2][]{\frac{\partial #1}{\partial #2}}
\newcommand{\code}[1]{\texttt{#1}}


%Titel,Inhalt----------------------------------------------------------------------------------------------------------

\begin{document}
\pagenumbering{gobble} %verstecke Seitenzahl
\maketitle
\newpage

\section*{Abstract}
Das Rastertunnelmikroskop ist ein Bildgebungsinstrument zur Auflösung von atomaren Strukturen. Mit einem optischen Mikroskop lassen sich aufgrund der Wellenlängen des sichtbaren Lichtes nur Auflösungen bis ca. 200nm erreichen. Da Atomdurchmesser sich jedoch in der Größenordnung von $x=1\,\mathrm{\AA}=0.1\,\mathrm{nm}$ befinden, bedarf es hierzu einer deutlich besseren Auflösung. Das Rastertunnelmikroskop erreicht diese (im Bereich von $1\,\mathrm{\AA}$) durch Abfahren der Oberfläche in kleinen Schritten mit einer Spitze und Messen des Abstands durch Ausnutzen des quantenmechanischen Tunneleffekts.\\

Mit dem Rastertunnelmikroskop lassen sich nur leitende Materialien untersuchen, da die Messung über einen Tunnelstrom zwischen Probe und Spitze funktioniert. Mit dem Rastertunnelmikroskop "`sieht"' man die Dichte der leitenden Elektronen eines Materials. In diesem Versuch wollen wir mit Hilfe eines Rastertunnelmikroskops Graphit, Molybdändisulfid und Gold untersuchen. Aus den Bildern für Graphit berechnen wir die Gitterkonstante des Graphitkristalls. Die weiteren Ergebnisse werten wir qualitativ aus, um die Funktionsweise des Rastertunnelmikroskops zu verstehen. Außerdem untersuchen wir die Bilder auf Artefakte, die durch störende Einflüsse von außen oder Verunreinigungen und Fehler der Spitze oder des Materials entstehen.\\

Zum Untersuchen der Proben stellen wir die Messspitzen aus Platin-Iridium-Draht her. Es gelangen uns zwei Spitzen, mit denen wir die atomare Struktur von Graphit messen konnten. Allerdings konnten wir damit in Molybdän keine Struktur erkennen, weshalb eine Messung von Gold uns nicht möglich war.\\

Als Ergebnis für die Gitterkonstante von Graphit erhielten wir:
\begin{align*}
	a=\text{HIER ERGEBNIS EINSETZEN}
\end{align*}

\newpage

\thispagestyle{empty}
\tableofcontents
\newpage

%Schreiben----------------------------------------------------------------------------------------------------------
\pagenumbering{arabic} %verstecke Seitenzahl
\section{Einleitung}

Das Rastertunnelmikroskop dient der Mikroskopie von leitenden Materialien bei atomarer Auflösung. Diese wird erreicht, indem eine Messspitze bis auf wenige $\mathrm{\AA}$ Abstand an die Probe bewegt und zwischen Spitze und Probe eine Spannung angelegt wird. Durch den quantenmechanischen Tunneleffekt fließt nun zwischen Spitze und Probe ein abstandsabhängiger Tunnelstrom. Die Spitze wird nun in einem Raster über die Probe bewegt und die Probe dadurch abgetastet.\\

In diesem Versuch untersuchen wir die Eigenschaften eines Rastertunnelmikroskops, indem wir damit drei Proben (Graphit, Molybdändisulfid und Gold) vermessen. Ein Rastertunnelmikroskop erreicht Auflösungen von bis zu ca. $1\,\mathrm{\AA}$, womit sich bei Optimalbedingungen die Atomstruktur dieser Materialien auflösen lässt. Wir versuchen in diesem Versuch, möglichst gute Spitzen herzustellen und die Parameter des Rastertunnelmikroskops optimal einzustellen, um möglichst gute Bilder der Proben zu erhalten.



\newpage
\section{Theoretische Grundlagen}

Zum Verstehen dieses Versuchs empfiehlt es sich, gewisse theoretische Grundlagen verstanden zu haben. Im Folgenden finden sich die wichtigsten Grundlagen für diesen Versuch. Dieser Teil orientiert sich weitgehendst an \cite{staat}.

\subsection{Quantenmechanischer Tunneleffekt}\label{tunnel}

Wenn ein Teilchen der Energie $E$ auf eine Potentialbarriere trifft mit dem Potential $E_0$ mit $E_0>E$, wird das Teilchen im klassischen Modell komplett reflektiert und kann die Potentialbarriere nicht überwinden. In der Quantenmechanik ist der Ort des Teilchens nun nicht mehr eindeutig festgelegt, sondern wird durch eine Aufenthaltswahrscheinlichkeit $|\Psi|^2$ beschrieben. Die Wellenfunktion $\Psi$, welche die Aufenthaltswahrscheinlichkeit festlegt, wird für ein freies Teilchen im Allgemeinen durch eine Sinus-Funktion beschrieben. Die Aufenthaltswahrscheinlichkeit wird nun am Rand der Potentialbarriere nicht komplett null, sondern nimmt exponentiell ab. Bei genügend kleiner Breite $d$ der Potentialstufe besteht nun noch eine endliche Aufenthaltswahrscheinlichkeit für die Teilchen auf der anderen Seite der Potentialbarriere. Dies bedeutet, dass die Teilchen mit einer gewissen Wahrscheinlichkeitsdichte die Potentialbarriere überwinden können. Diesen Effekt nennt man Tunneleffekt. \textsuperscript{\cite{demtroeder3}}

\gra{tunneleffekt}{Tunneln eines Teilchens mit Wellenfunktion \textcolor{red}{\textbf{$\Psi$}} am Rechteckpotential \textcolor{blue}{\textbf{$E_0$}} der Breite $d$. (Selbsterstellte Grafik, orientiert an \cite{demtroeder3})}

\subsection{Tunnelstrom}

Wird die Spitze des Rastertunnelmikroskops in einen Abstand von $d\approx1\,nm$ zur Probe gebracht und eine Spannung zwischen Spitze und Probe angelegt, fließt durch den quantenmechanischen Tunneleffekt (siehe Kapitel \ref{tunnel}) nun ein Tunnelstrom zwischen Spitze und Probe, der exponentiell vom Abstand $d$ abhängt.

\subsection{Piezoelektrischer Effekt}\label{piezo}

In piezoelektrischen Kristallen sind die Ionen so angeordnet, dass die Gesamtladung $Q=0$ ist. Wenn der Kristall deformiert wird, verschieben sich die Ionen und die Ladungsschwerpunkte im Kristall. Dadurch entsteht eine Spannung zwischen beiden Seiten des Kristalls, die linear von der deformierenden Kraft abhängt (siehe Abbildung \ref{piezobild}). Dies nennt sich piezoelektrischer Effekt.\\

\gra[0.6]{piezo}{Veranschaulichung des piezoelektrischen Effekts\textsuperscript{\cite{piezo}}\label{piezobild}} 

Ebenso gibt es den inversen piezoelektrischen Effekt. Eine von außen angelegte Spannung verändert die Anordnung der Ionen und bewirkt so eine Deformation des Kristalls. Durch die Spannung kann der Kristall ausgedehnt oder gestaucht werden.

\subsection{Funktionsweise des Rastertunnelmikroskops}

\label{funktionsweise}
\gra{rtm}{Rastertunnelmikroskop\label{rtm}\textsuperscript{\cite{anleitung}}}

Das Rastertunnelmikroskop funktioniert über eine Platin-Iridium-Spitze, welche über ein Piezo-Element (siehe Kapitel \ref{piezo}) in $z$-Richtung bis auf wenige $\mathrm{\AA}$ an die Probe angenähert wird (siehe Abbildung \ref{rtm}). Zwischen Spitze und Probe wird eine Spannung angelegt, welche einen Tunnelstrom zwischen Spitze und Probe erzeugt (siehe Kapitel \ref{tunnelstrom}). Aus der Stärke des Tunnelstroms kann nun die Entfernung zwischen Probe und Spitze bestimmt werden. Mit zwei weiteren Piezo-Elementen in $x$- und $y$-Richtung wird die Spitze schrittweise über die Probe bewegt und damit das Oberflächenprofil der Probe aufgenommen. Das Rastertunnelmikroskop besitzt außerdem einen Schrittmotor, welcher die Entfernung zwischen Spitze und Probe in groben Schritten verändern kann.

\paragraph{Constant Height Mode}

Im Constant Height Mode verändert sich die Position der Spitze in $z$-Richtung nicht. Die Probe wird in $x$- und $y$-Richtung gerastert und aus dem Tunnelstrom der Abstand und damit das Oberflächenprofil der Probe berechnet. Der Vorteil dieses Verfahrens ist, dass eine schnellere Messung möglich ist. Allerdings funktioniert das Verfahren nur bei sehr flachen Oberflächen, da die Spitze ansonsten mit der Probe kollidieren könnte.

\paragraph{Constant Current Mode}

Im Constant Current Mode wird die Position der Spitze so nachgeregelt, dass der Tunnelstrom und damit der Abstand der Spitze zur Probenoberfläche konstant ist. Das Oberflächenprofil entspricht nun der Bewegung der Spitze in $z$-Richtung. Bei diesem Verfahren ist nun auch eine Messung von weniger flachen Oberflächen möglich, allerdings braucht es mehr Zeit, da die Nachregelung des Signals ansonsten nicht schnell genug nachkommt.\\

In diesem Versuch verwenden wir den Constant Current Mode, um die Gefahr der Kollision der Spitze mit der Probe gering zu halten. 

\subsection{Bändermodell}

\gra{baender}{Darstellung des Bändermodells für Metalle, Halbleiter und Isolatoren\textsuperscript{\cite{demtroeder3}}\label{baender}}

Im Bändermodell werden die Elektronen je nach Energie verschiedenen Bändern zugeordnet. Dabei sind Bänder oberhalb der Fermienergie $E_F$ Leitungsbänder und Bänder unterhalb der Fermienergie $E_F$ Valenzbänder (siehe Abbildung \ref{baender}). \\

Bei Metallen liegt die Fermi-Energie innerhalb eines Bandes und unterteilt dies somit in Valenz- und Leitungsband. Der Abstand zwischen Valenz- und Leitungsband bei Nichtmetallen nennt sich Bandlücke. Bei Halbleitern ist diese Bandlücke klein, sodass sie durch thermische Anregungen überwunden werden und der Halbleiter so in den leitenden Zustand übergehen kann. Bei Isolatoren reicht die Bandlückenenergie bereits aus, um einen Phasenwechsel des Materials durchzuführen. Ein Isolator ist also bei keiner Temperatur leitend.\\

Halbmetalle gehören zu den Halbleitern, haben jedoch eine sehr kleine Bandlücke ($\sim k_BT$), sodass sie bereits bei Raumtemperatur leitend sind. Anders als bei Metallen steigt die Leitfähigkeit jedoch mit der Temperatur.

\subsection{Materialien unter dem Rastertunnelmikroskop}

\subsubsection{Graphit}
\gra{graphit}{Kristallstruktur von Graphit\textsuperscript{\cite{anleitung}}\label{graphitstruktur}}

Graphit ist ein Halbmetall und besitzt eine hexagonale Kristallstruktur in mehreren Ebenen, wobei die Ebenen zueinander versetzt sind (siehe Abbildung \ref{graphitstruktur}). Dabei werden die Elektronen, die ein direkt benachbartes Atom in der darunterliegenden Ebene haben, als $\alpha$-Atome bezeichnet und die übrigen als $\beta$-Atome. Da die Elektronen eines $\alpha$-Atoms auch mit dem darunterliegenden $\alpha$-Atom koppeln, tragen hauptsächlich die Elektronen der $\beta$-Atome zum Tunnelstrom bei. Deshalb sind im Wesentlichen nur die $\beta$-Atome im Elektronenmikroskop sichtbar.

\subsubsection{Molybdändisulfid}
\gra{molybdaen}{Struktur von Molybdändisulfid\textsuperscript{\cite{mol1}\cite{mol2}\label{molybdaen}}}
Molybdändisulfid ($\mathrm{MoS}_2$) hat die in Abbildung \ref{molybdaen} gezeigte Struktur und ist ein Halbleiter. Deshalb muss die Abtastspannung bei Molybdändisulfid hoch genug sein, um die Bandlückenenergie zu überwinden und die Elektronen des Molybdändisulfids in das Leitungsband anzuregen. 

\subsubsection{Gold}
\gra{gold}{Kristallstruktur von Gold\textsuperscript{\cite{kittel}\label{gold}}}
Gold ist ein Metall und ist also immer leitend. Die Kristallstruktur ist in Abbildung \ref{gold} zu sehen. Allerdings ist Gold unter dem Rastertunnelmikroskop nur sehr schwer zu erkennen, da es in Luft sehr leicht verunreinigt wird. Um gute Bilder von Gold zu erhalten, müsste man das Rastertunnelmikroskop im Vakuum betreiben, was in diesem Versuch jedoch nicht gemacht wird.


\newpage
\section{Versuchsaufbau}\label{aufbau}
\gra{aufbau}{Schematische Zeichnung des Versuchsaufbaus\textsuperscript{\cite{demtroeder3}}}
Der Versuch besteht aus einem Rastertunnelmikroskop, der dazugehörigen Steuerungseinheit und einem Computer. Das Rastertunnelmikroskop selbst besteht, wie in Kapitel \ref{funktionsweise} beschrieben, aus drei Piezo-Elementen und der Spitze. Außerdem befindet sich über dem Rastertunnelmikroskop ein Glasdeckel mit Lupe, um die Spitze beobachten zu können. Auf dem Computer läuft die Steuerungssoftware \emph{EasyScan 2} des Rastertunnelmikroskops.

\newpage
\section{Durchführung}\label{durchfuehrung}
Um ein Bild mit dem Rastertunnelmikroskop aufzeichnen zu können, benötigen wir eine Messspitze. Diese stellen wir aus Platin-Iridium-Draht her. Dazu benutzen wir eine Zange, mit der wir den Draht festhalten und eine weitere, mit der wir in einem möglich spitzen Winkel den Draht abreißen, um eine möglichst feine Spitze herzustellen. Da es mit diesem Verfahren schwierig ist, eine optimale Spitze herzustellen, erwarten wir viele Artefakte, z.B. durch schlechte Spitzen. Ebenso tragen Verschmutzungen der Spitze oder der Probe zu möglicher Artefaktbildung bei.\\

Die so hergestellte Messspitze und die zu betrachtende Probe legen wir in das Rastertunnelmikroskop ein. Wir beginnen mit der Graphitprobe, da diese am einfachsten zu messen ist. Nun führen wir die Probe mit dem Schrittmotor so nah an die Spitze heran, wie man mit Hilfe der Lupe noch erkennen kann, dass die Spitze die Probe nicht berührt. Wenn die Spitze die Probe berührt wird sie vermutlich durch den Kontakt zerstört. Durch das Computerprogramm erfolgt nun die Feinannäherung der Spitze an die Probe. \\

Ist die Feinnäherung erfolgt, beginnt das Programm, die Probe mit der Spitze abzurastern und das Oberflächenprofil zu erstellen. Die Bildgröße lässt sich hierbei einstellen. Zu Anfang stellen wir die Bildgröße immer auf $500\,$nm Breite und Höhe. Dann suchen wir uns einen homogenen Bereich  aus und nehmen diesen auf. Wenn kein farbiges Bild erzeugt wird, hat die Feinannäherung keinen Erfolg gehabt, was möglicherweise an der Qualität der Spitze liegt. Dann probieren wir die Feinannäherung erneut und verwerfen nach einigen Fehlversuchen die Messspitze.\\

Wir wiederholen dieses Verfahren so oft, bis wir die Atomstruktur von Graphit erkennen können. Zur Verbesserung der Bilder lassen sich die Einstellungen der Steuerelektronik anpassen. Wenn wir nun ein gutes Bild für Graphit erhalten haben, tauschen wir die Graphitprobe durch Molybdändisulfid und anschließend Gold aus und versuchen, für diese Materialien eine atomare Auflösung zu erhalten.
\newpage
\section{Auswertung}



\subsection{Artefakte}

\subsubsection{Schwarze und weiße Ränder}

Ein Artefakt, das wir häufig beobachteten waren schwarze bzw. weiße Ränder. Dabei ist ein weißer (bzw. schwarzer Rand) an der linken Seite der Topograpie der Scan-forward Richtung zu sehen, und ein schwarzer (bzw. weißer Rand) auf der rechten Seite der Topographie der Scan-backwards Richtung, wie z.B in Abbildung \ref{BW}.\\

\gra[0.8]{Black-white-edge-Graphit38}{Schwarze und weiße Ränder durch Aufsammeln eines Teilchens \label{BW}}

Dieser Effekt kommt zu Stande, wenn mit der Spitze ein Teilchen, z.B. ein Molekül aufgesammelt wurde. Ähnlich wie beim Hin- und Herziehen eines Stiftes über ein Papier, kippt dieses Teilchen bem Ändern der Scanrichtung um, sodass es kurzzeitig zu fehlerhaften Strömen kommt.
Prinzipiell ist das aufgesammelte Teilchen nicht immer schlecht, manchmal führt erst dieses zu einer atomaren Auflösung. In der Forschung werden teilweise absichtlich Moleküle bestimmter Stoffe aufgesammelt um eine bessere Auflösung zu erhalten.
Durch einen Cleaning Pulse kann versucht werden, die Spitze zu reinigen und das Teilchen "abzuschütteln". Es empfiehlt sich dann den Scanbereich zu wechseln, um dieses Teilchen nicht direkt wieder aufzusammeln.

\subsubsection{Symmetrie}

Auf einigen Aufnahmen konnten wir Achsensymmetrien feststellen, wobei die Symmetrieachse in der Mitte des Bildes liegt (siehe Abbildung ). Vermutlich kommen diese Symmetrien durch eine fehlerhafte Führung der Spitze zustande. Es liegt nahe anzunehmen, das die Spitze nach der Hälfte des Bildes in die andere Richtung zurück scannt. 

\gra[0.6]{Symmetry-Graphit18}{Beispiel der beobachteten Symeetrien}

Auffällig ist hierbei, dass wir diese Symmetrien nur auf großen Scalen beobachten konnten, also bei Bildern von 500 nm Auflösung. Das Problem bezüglich der Spitzenführung scheint also nur bei größeren abzutastenden Flächen zu entstehen.

\subsubsection{Verzerrung \label{Scaling}}
Ein anderes Artefakt, das wir auf einigen Aufnahmen beobachten konnten, ist eine gewisse Verzerrung bei den verschiedenen Scanrichtungen. In Abbildung \ref{Scal} ist klar zu erkennen, das einzelne Strukturen bei der Scanforward-Richtung andere Längen aufweisen als in der Aufnahme mit entgegengesetzter Scan-Richtung. So ist z.B. das am rechten Hand zu sehende kreisförmige Struktur im rechten Bild deutlich kleiner und nahezu kreisförmig wohingegen im rechten Bild eine längere ovale Struktur zu sehen ist.


\gra[0.6]{Scaling-Graphit20}{Verzerrungen in den Aufnahmen \label{Scal} }

Wir erklären uns diese Effekte durch einen nicht konstante Scangeschwindigkeit. Vermutlich ist die Geschwindigkeit mit der sich die Spitz über die Probe bewegt nicht immer konstant, sondern variiert teilweise, was zu einer solchen Verzerrung führen kann.

\subsubsection{Nicht-weißes Rauschen}
Während der Durchführung des Versuchs beobachteten wir außerdem öfters Rauschen. Einen dieser Störeffekte konnten wir als nicht-weißes Rauschen identifizieren und haben dieses analysiert und die Frequenz berechnet.
Wir beobachteten dabei Nicht-weißes Rauschen unterschiedlicher Frequenzen.
\paragraph{Nieder-frequentes Rauschen}
Zu Begunn des Versuchs am ersten Versuchstag beobachteten wir offenbar ein Rauschen konstanter Frequenz, wie in Abbildung \ref{NWR} dargestellt.

\graThree{Graphit2}{Rauschen1-Bild-Graphit2}{Rauschen1-graph-Graphit2}{Nider-frequentes Nicht-weißes Rauschen \label{NWR}}

In Abbildung \ref{NWR} ist links die urpsrüngliche Aufnahme dargestellt. In der Mitte sind die verschiedenen Linien entlang welcher wir das Profil auslesen abgebildet. Rechts sind man schließlich diese Profile. Wir wählen dabei drei verschiedene Linien an veschiedenen Stellen, um zu überprüfen ob es sich tatsächlich um ein Rauschen konstanter Frequenz handelt. Wir sehen rechts in Abbildung \ref{NWR}, das das Rauschen immer jeweils 17 Perioden (Anzahl der Perioden N) pro Linie hat, wir berrechnen also die Frequenz f bei einer Scangeschwindigkeit $v$ von 0.2sec/Linie:

\begin{align*}
f=\frac{N}{v}=85 Hz
\end{align*}

Den Fehler schätzen wir aus dem abgeschätzten Fehler auf die Anzahl der Perioden ab, den wir mit $s_N = \pm 1$ abschätzen. Damit erhalten wir:
\begin{align*}
f=(85\pm 5) Hz
\end{align*}


Eine Frequenz von 85 Hz ist sehr untypisch, erwartet hätten wir eine Frequenz von 50 Hz bzw. 60 Hz, die der Stromversorgung zugeordnet werden könnte. Recherche ergab, das einige Monitore mit einer Frequenz von 85 Hz laufen, ein Überprüfen der Einstellungen des im Versuch verwendeten Monitors ergab jedoch, das dieser mit 60 Hz betrieben wird. Wir können uns die Ursache dieses Rauschen also erst einmal nicht erklären. Möglich wäre, dass die Angabe der Scangeschwinidgkeit nicht stimmt. Wie in Abschnitt \ref{Scaling} erklärt, scheint diese Schwankungen zu unterliegen. Unter der Annahme einer fehlerhaften Scangeschwindigkeit, wäre das beobachtete Rauschen möglicherweise dann doch der Stromversorgung zuzurechnen. 

\paragraph{hochfrequentes Rauschen}

Am zweiten Versuchstag beobachteten wir erneut nicht-weißes Rauschn, jedoch mit einer deutlich höheren Frequenz. Ohne an den Einstellungen etwas zu verändern erhielten wir nach einigen Minuten ein rauschfreies Bild mit atomarer Auflösung.
Dadurch, dass dieses Rauschen deutlich hochfrequentiger ist, ist schwerer zu erkennen, ob es sich tatsächlich um ein Rauschen konstanter Frequenz handelt. Um dies zu überprüfen betrachten wir also beide Scanrichtungen.

\graTwo[0.45]{420Hz-topoforward}{420Hz-Graph-forward}{Hochfrequentes Nicht-weißes Rauschen - Scan-forward Richtung \label{NRW1}}


\graTwo[0.45]{420Hz-topofbackward}{420Hz-Graph-backwards}{Hochfrequentes Nicht-weißes Rauschen - Scan-backward Richtung \label{NRW2}}

An Hand der ausgelesenen Profile rechts in Abbildung \ref{NRW1} und \ref{NRW2} erkennen wir, dass die Amplitude zwar stark schwankt, die Periodendauer jedoch konstant scheint.
Wir berechnen wie zuvor die Frequenz f des Rauschens:


\begin{align*}
f=\frac{N}{v}=\frac{85}{0.2s^{-1}}= (420 \pm 5) Hz
\end{align*}

wobei sich der Fehler analog wie zuvor berechnet, auch hier schätzten wir  $s_N = \pm 1$.
Auffällig ist bei diesem Ergebnis, dass es ein Vielfaches des Niederfrequenten Rauschen ist.


\subsubsection{Fehlschlagen des Approach-Vorgangs}

Ein weiteres Artefakt, das wir häufig beobachten konnten, war eine schwarz-weiße Topographie, die ensteht wenn ein Approach-Vorgang fehl schlägt, also z.B. die Spitze bei diesem Vorgang zerstört wird (siehe Abbildung \ref{AF})

\gra[0.6]{Graphit4_Approach_fail}{Fehlschlagen eines Approach-Vorgangs \label{AF}}

In einem solchen Fall kann der Approach-Vorgang wiederholt werden, wurde die Spitze allerdings tatsächlich zerstört hilft nur die Herstellung einer neuen Spitze.

\subsection{Atomare Auflösung für Graphit}

Wir konnten für Graphit mit zwei verschiedenen Spitzen atomare Auflösung erreichen. Mit einer ersten Spitze erhöhten wir den Auflösung schrittweise, sodass wir eine Reihe von Bildern mit klaren Strukturen bei verschiedenen Auflösungen erhielten (siehe Abbildung \ref{GA} )

\graFive[0.35]{Graphit47nm}{Graphit20nm}{Graphit8nm}{Graphit3nm}{Graphit1nm}{Atomare Auflösung von Graphit bei verschiedenen Abständen \label{GA}}

In Abbildung \ref{GA} ist dabei zu erkennen, dass besonders bei der ersten Aufnahme mit einer Auflösung von 47 nm noch Artefakte auftreten, die vermutlich durch Nachführschwierigkeiten auftreten oder tatsächlich durch ein Rauschen verursacht wurden.\\

Mit einer zweiten Spitze erhielten wir eine noch bessere Auflösung. Wir nahmen hier Bilder mit einer Auflösung von 4 nm, 3nm und 1nm auf. Dies ist in Abbildung \ref{GA2} dargestellt.

\vskip 20 pt
\graThree{Graphit2-4nm}{Graphit2-3nm}{Graphit2-1nm}{Beste erreichte Auflösung für Graphit \label{GA2}}

\subsubsection{Berechnung der Gitterkonstante}
An Hand der Aufnahme aus diesem Set mit einer Auflösung von 2.5 nm berrechnen wir die Gitterkonstante von Graphit. Dazu verwenden wir das Programm Gwyddion.
Dieses erlaubt es, Linien durch die erhaltenen Aufnahme zu legen, an der das Profil ausgelesen werden kann. An den Peaks fitten wir dann Gaußfits aus deren Position wir die Abstände der Atome und somit die Gitterkonstante berechnen können. Um Unregelmäßigkeiten der Gitterstruktur sowie Unregelmäßigkeiten die durch das Abtasten enstehen können auszugleichen, werten wir mit Hilfe von drei verschiedenen Profillinien aus: Einer horizontalen Linie sowie einer von links oben nach rechts unten verlaufenden Linie und einer von rechts oben nach links unten verlaufenden Linie. Die zuvor beschriebene Verzerrung (siehe \ref{Scaling}) konnten wir hier nicht beobachten, sodass die Auswertung der Topographie einer Scanrichtung ausreicht.

\paragraph{Horizontales Profil}

In Abbildung \ref{HP} ist links die Linie eingezeichnet, entlang welcher das Profil ausgewertet wird. Dieses ist rechts mit den gefitteten Gaußfunktionen dargestellt. 
\vskip 10 pt

\graTwo[0.45]{Bild2D54}{Graphit-horizontal}{Horizontales Profil von Graphit \label{HP}}

Die einzelnen Fits haben die Form:

\begin{align*}
f(x) = y_0 + A\cdot exp\left[ -\frac{(x-\mu)^2}{\sigma^2}\right] 
\end{align*}

Die Parameter der einzelnen Fits sind in Tabelle \ref{GT} dargestellt.
Da der Parameter $y_0$ für die Auswertung keinerlei Relevanz hat, wird dieser nicht aufgelistet

\begin{table}[h!]
	\centering
	\begin{tabular}{c|c|c|c}
		Peak & A & $\mu$ & $\sigma$ \\ \hline
		Peak 1 & $(42 \pm 6 )pm$ & $(0.167 \pm 0.002)nm$ & $(161\pm20) pm$\\
		Peak 2 & $(57 \pm 19 )pm$ & $(0.5122 \pm 0.0019)nm$ & $(194\pm43) pm$\\
		Peak 3 & $(34 \pm 4 )pm$ & $(0.848 \pm 0.002)nm$ & $(122\pm14) pm$\\
		Peak 4 & $(29.4 \pm 1.8 )pm$ & $(1.155 \pm 0.003)nm$ & $(109\pm8) pm$\\
		Peak 5 & $(32.4 \pm 1.6 )pm$ & $(1.503 \pm 0.003)nm$ & $(114\pm8) pm$\\
		Peak 6 & $(22.6 \pm 1.9 )pm$ & $(1.793 \pm 0.003)nm$ & $(105\pm12) pm$\\
		Peak 7 & $(27 \pm 3 )pm$ & $(2.125 \pm 0.004)nm$ & $(97\pm13) pm$\\
	\end{tabular}
	\caption{Gauß-Peaks des horizontalen Profils \label{GT}}
\end{table}

\paragraph{Diagonales Profil (links oben nach rechts unten)}

Da Diagonal eine geringere Anzahl an Atomen gut erkennbar ist, deckt die hier gezogene Linie nur einen Teil des Ausschnitts ab, um noch ein erkennbares Profil zu erhalten.

\vskip 10 pt

\graTwo[0.45]{Bild2Ddiagonal}{Gaussfit-diagonalloru}{Diagonales Profil (links oben nach rechts unten)}

Man erkennt dabei, das auch bei einer Auswahl von nur vergleichsweise sehr gut erkennbaren Atomen das Profil deutlich schlechter ist, als beim Durchlegen einer horizontalen Linie.
Da sich bei der Auswertung des horizontalen Profils gezeigt hat, dass das Auswerten vom ersten und letzten Peak zu Berechnung des atomaren Abstandes ausreicht, werden hier nur diese aufgeführt.
\begin{table}[h!]
	\centering
	\begin{tabular}{c|c|c|c}
		Peak & A & $\mu$ & $\sigma$ \\ \hline
		Peak 1 & $(203 \pm 36000 )nm$ & $(155 \pm 7)pm$ & $(13.6\pm13000) pm$\\
		Peak 5 & $(815.65 \pm 0.15 )pm$ & $(1267 \pm 4)pm$ & $(739.8\pm6800) pm$\\
	\end{tabular}
	\caption{Gauß-Peaks des diagonalen (links oben - rechts unten) Profils \label{GT1}}
\end{table}


\paragraph{Diagonales Profil (rechts oben nach links unten)}

In dieser Richtung entschieden wir uns für eine Linie mit lediglich vier Peaks, dies vereinfacht das Identifizieren der Atome im ausgelesenen Profil und wählen der passenden Fitbereiche.

\graTwo[0.45]{Bild2Ddiagonal2}{Gaussfit-alle-diagonal2}{Diagonales Profil (rechts oben nach links unten)}

\begin{table}[h!]
	\centering
	\begin{tabular}{c|c|c|c}
		Peak & A & $\mu$ & $\sigma$ \\ \hline
		Peak 1 & $(131 \pm 570 )pm$ & $(107 \pm 3)pm$ & $(227\pm530) pm$\\
		Peak 4 & $(18 \pm 2 )pm$ & $(720 \pm 3)pm$ & $(40\pm7) pm$\\
	\end{tabular}
	\caption{Gauß-Peaks des diagonalen (links oben - rechts unten) Profils \label{GT2}}
\end{table}

\subsection{Molybdändisulfid}

Sofern wir auf Graphit mit einer Spitze atomare Auflösung erreichen konnten, versuchten wir mit dieser Spitze atomare Auflösung auf Molybdänsulfid zu erreichen. Dies gelang jedoch leider nicht. Das beste Bild, das wir auf Molybdänsulfid erzielen konnten, ist in Abbildung \ref{MS} dargestellt.

\gra[0.6]{Mo7}{Aufnahmen von Molybdänsulfid \label{MS}}


\subsection{Gold}

Da wir bereits für Molybdändisulfid keine atomare Auflösung erreichen konnten, haben wir für Gold keine Bilder aufgenommen, da dies nur sinnvoll ist, wenn man eine auf Molybdändisulfid gut auflösende Spitze hat. 

\newpage
\section{Zusammenfassung und Diskussion}
\label{diskussion}

%\newpage
%\section{Anhang}

%\subsection{Grafiken}\label{bilder}




%\subsection{Tabellen}\label{tabellen}


%\newpage
%\subsection{Quellcode (MATLAB)}
%\lstinputlisting[language=MATLAB]{Rohdaten/alpha.m}
%\clearpage
%\subsection{Quellcode (R)}
%\subsubsection{\code{auswertung.R}}\label{auswertungR}
%\lstinputlisting[language=R]{r-files/auswertung.R}

%\subsection{Laborheft}\label{laborbuch}
%\begin{minipage}{\textwidth}
%	\centering
%	\includegraphics[width=0.9\textwidth]{laborbuch/laborbuch6.pdf}
%\end{minipage}
\newpage
\listoffigures

%Literatur----------------------------------------------------------------------------------------------------------

%\cite{les}
\newpage
\thispagestyle{empty}
\begin{thebibliography}{9}

%\bibitem{staat}
%  Tobijas Kotyk,
%  \emph{Versuche zur Radioaktivität im Physikalischen Fortgeschrittenen Praktikum an der Albert-Ludwigs-Universität Freiburg},
%  Albert-Ludwigs-Universität, Freiburg,
%  2005
  

  
%\bibitem{molmasse}
%  \emph{http://www.convertunits.com/molarmass/<ELEMENTNAME AUF ENGLISCH>}, Stand 28.09.2015
  

\bibitem{anleitung}
M. Köhli, S. Röttinger,\\
\emph{Versuchsanleitung Fortgeschrittenen Praktikum: Rastertunnelmikroskop},\\
Albert-Ludwigs-Universität Freiburg,\\
2013

\bibitem{staat}
Dieter Ritzmann,\\
\emph{Einrichtung eines Versuchs: Rastertunnelmikroskopie für das Fortgeschrittenenpraktikum 2},\\
Albert-Ludwigs-Universität Freiburg,\\
1995

\bibitem{demtroeder3}
Wolfgang Demtröder,\\
\emph{Experimentalphysik 3 - Atome, Moleküle und Festkörper},\\
Springer Spektrum,\\
5. Auflage, 2016

\bibitem{mol1}
\emph{http://www.a-m.de/images/molybdaendisulfid\_01grd.jpg},\\
Stand: 13.10.2016

\bibitem{mol2}
N.N. Gmelin,\\
\emph{Handbuch der Anorganischen Chemie},\\
Springer Verlag Berlin, Heidelberg, New York,\\
8. Auflage, 1992

\bibitem{kittel}
Ch. Kittel,\\
\emph{Einführung in die Festkörperphysik},\\
Oldenbourg Verlag München,\\
14. Auflage, 2006

\bibitem{piezo}
\emph{http://www.piezoeffekt.de/2crystals.php},\\
Stand: 13.10.2016

\end{thebibliography}

\end{document}